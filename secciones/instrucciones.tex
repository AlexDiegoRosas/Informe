\begin{figure}
    \includegraphics[width=0.2\textwidth]{img/senati.png}
\end{figure}
\section*{\centering\textbf{INSTRUCCIONES PARA EL USO DEL CUADERNO DE INFORMES}}
\section{\normalsize{ PRESENTACIÓN.}}

El Cuaderno de Informes de trabajo semanal es un documento de control, en el cual el estudiante,
registra diariamente, durante la semana, las tareas, operaciones que ejecuta en su formación
práctica en SENATI y en la empresa.

\section{\normalsize{INSTRUCCIONES PARA EL USO DEL CUADERNO DE INFORMES}}
\begin{itemize}
\item [2.1] {En el cuadro de rotaciones, el estudiante, registrará el nombre de las áreas o secciones
            por las cuales rota durante su formación práctica, precisando la fecha de inicio y término.}
\item [2.2] {Con base al PEA proporcionado por el instructo, el estudiante transcribe el PEA en el 
            cuaderno de informes. El estudiante irá registrando y controlando su avance, marcando en la
            columna que corresponda.}
\item [2.3] {En la hoja de informe semanal, el estudiante registrará diariamente los trabajos que ejecuta,
            indicando el tiempo correspondiente. El día de asistencia al centro para las sesiones de tecnología,
            registrará los contenidos que desarrolla. Al terminó de la semana totalizará las horas.
            
            De las tarea ejecutadad durante la semana, el estudiante seleccionará la más
            significativa y hará una descripción del proceso de jecución con esquemas y dibujos
            correspondientes que aclaren dicho proceso.}
\item [2.4] {Semanalmente, el estudiante registrará su aistencia, en los casilleros correspondientes.}
\item [2.5] {Semanalmente, el Monitor revisará, anotará las observaciones y recomendaciones que considere, el 
            intructor revisará y calificará el Cuaderno de Informes haciendo las observaciones y recomendaciones
            que considere convenientes, en los aspectos relacionados a la elaboración de un Informe Técnico (términos
            técnicos, dibujo técnico, descripción de la tarea y su procedimiento, normas técnicas, seguridad, etc.)}
\item [2.6] {Si el PEA tiene menos operaciones $(151)$ de las indicadas en el presente formato, puede eliminar alguna
            página. Asimismo, para el informe de las semanas siguientes, debe agregar las semanas que corresponda.}
\item [2.7] {Escala de calificación vigesimal}
\end {itemize}

\begin{table}[!ht]
    \centering
    \begin{tabular}{| c | c | c |}
        \hline
        \rowcolor{gray!30}
        \textbf{CUANTITATIVA} & \textbf{CUALITATIVA} & \textbf{CONDICIÓN}        \\ \hline
        $16.8-20.0$           & Excelente            & \multirow{3}{*}{Aprobado} \\ \cline{1-2}
        $13.7-16.7$           & Bueno                &                           \\ \cline{1-2}
        $10.5-13.6$           & Aceptable            &                           \\ \hline
        $00-10.4$             & Deficiente           & Desaprobado               \\ \hline
    \end{tabular}
    \caption{Cuadro de calificaciones}
    \label{tab:calificaciones}
\end{table}